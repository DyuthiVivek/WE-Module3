\documentclass{article}
\usepackage{hyperref}

\title{WE-Module3}
\author{Dyuthi Vivek}
\date{March 2024}


\begin{document}

\maketitle

\section{Introduction/Problem Statement}
To prompt GenAI to write code for the diamonds bidding game, develop a strategy for the game and make observations.

\medskip

\section{Teaching GenAI the game}
I prompted ChatGPT with the set of rules, and it claimed to have understood the game. When I asked it to write code, it misunderstood several aspects and implemented them wrongly. It did not provide one suite of cards to each player. It did not implement the scoring strategy in case of a tie. It created a new card called "T". There were several errors in the initial version of code. Upon providing several explanations, it was able to fix these. 

\medskip
Sometimes, GenAI very confidently bluffs and states that it has made the required changes without changing much. But generally, clear the specifications, better the response generated.

\medskip

\section{Iterating on strategy}
The initial strategy that ChatGPT came up with was actually very detailed, it considered lots of factors. But there were some obvious mistakes. It bid very high cards for low-valued cards in the very beginning. Apart from that, the code generated gave an IndexError several times. 

\medskip

The strategy was refined iteratively by prompting ChatGPT to incorporate some additional considerations, such as considering what cards other players have played, retaining high-value cards to be used later, to adapt bidding decisions dynamically, taking into consideration the number of rounds that have passed and the value of cards that have not been bid for yet. The final strategy that came up was good.

\medskip

\section{Analysis and Conclusion}

Some of the problems I faced while using GenAI:
I felt that the initial code generated by ChatGPT is not very "clean" and modular. But after specifying this, it makes changes.
Sometimes when a new response is generated, it undoes some other change I had requested earlier. 
Despite having clearly stated the specifications, it does not implement some of them unless they are re-iterated.
GenAI's lack of intuition and creativity makes it very different from humans.
\medskip 

Despite all of this, I find GenAI to be really impressive. I believe this will alter how we work in the future.

\section{Appendix}

Rules of the game:
It is played with three players. Each player gets a suit of cards other than the diamond suit. 
The diamond cards are then shuffled and put on auction one by one. 
All the players must bid with one of their own cards face down. 
The banker gives the diamond card to the highest bid, i.e. the bid with the most points.
2<3<4<5<6<7<8<9<T<J<Q<K<A
The winning player gets the points of the diamond card to their column in the table. If there are multiple players that have the highest bid with the same card, the points from the diamond card are divided equally among them.
The player with the most points wins at the end of the game.

\medskip
\medskip
Links: 

\begin{itemize}
    \item \href{https://docs.google.com/document/d/1Ux5MJa6416Uxm11PcR5k4KstW9Mxmtk88cdWXKjzT_M/edit?usp=sharing}{ChatGPT transcript}
    \item  \href{https://colab.research.google.com/drive/1xPR6YIhkXXMHtmyVIoWZa9e2pdipwR1P?usp=sharing}{Colab link}
\end{itemize}

 

\end{document}
